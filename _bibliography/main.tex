\documentclass{article}
\usepackage[utf8]{inputenc}
\usepackage{times}
\usepackage[numbers]{natbib}
% \usepackage[authoryear]{natbib}
\usepackage{url}  
\usepackage{xcolor}
\usepackage{hyperref}
\hypersetup{colorlinks=true,allcolors=blue}
\usepackage{hypcap}

\newcommand{\comment}[1]{\textit{#1}}
\newcommand*{\bla}[1]{\textit{\textcolor{olive}{#1}}}
%\newcommand{\bla}[1]{#1}
\newcommand{\todo}[1]{\textcolor{teal}{\textbf{TODO: } \texttt{#1}}}

\title{YearInReview}
\author{Mark Crowley}
\date{June 2021}

\begin{document}

\maketitle

\section{When does the year "begin" for you?}
For every field of human endeavour and job there seems to be a different answer for this question. Obviously, I don't mean the calendar year, there are many specific and exact answers to that (Gregorian: Jan 1, 2021, Chinese year XXX: Feb X, 2021, Persian Year XXX: May, 2021, Hebrew Year XXX: fdsjkfjl, ...). I mean the \textit{logical} year when you feel there is time and space to step back and take a look at where things stand and where they are going next.

\subsection{Logical Years}
Some common logical years that aren't calendar or culture specific are things like school year in North America, really starting up every September. The fiscal year for most businesses and governments is a hugely important logical year to such an extent that everyone needs to plan their vacations, trips, and other life activities around submitting final tax receipts on March 31 or the second quarter results on June 30.

In academia I wouldn't say there is one specific start of the logical year. We talk in school terms often, so September is important, again this is mostly a North American thing. But some universities and colleges have different numbers of terms, so it varies a lot.
In the AI/ML research field I think there's an argument for the start of a new year being somewhere around June 1, at least at the moment. 

With the new rearrangment of the major international AI/ML conferences about 6 or 7 years ago, IJCAI moved to spring/summer (northern hemisphere, sorry! I need to pick some context), AAAI moved to winter, after New Years (Gregorian). NeurIPS is still in December with a late spring deadline. So this moment right now, June 1, a bunch of major conferences finished (AAAI, IJCAI, ICML, ICLR) as well as important local conferences (CanAI, ???). We also have the deadlines for NeurIPS and ??? past. The next big deadlines are in the fall September (AAAI, AAMAS, ???).  Also, many researchers are done teaching, so they get the summer off...ha! no kidding. we never get any time off. ever. It's the time to frantically dive deeper into some research, to grasp some fleeting moment of productivity on the most important part of your job, \textbf{The Research}. Before the new students arrive in the fall, before you head off to that conference in August, before you need to teach in Sept or January.
Also, not coincidentally, it is the beginning of the season for tenure applications, job application, department and dean turnovers. It's the start of \textit{something new}. Isn't it exciting?

So what better time than "New Year" (Logical), to look back on the year that was?

\section{My Past Year in Research : Jan 2020 - May 2021}
\bla{I like to break down my research in the following way by primary method of approach}.
\begin{itemize}
    \item Abstraction - including \textbf{Dimensionality Reduction} and Manifold Learning
    \item Regression - including prediction, interpolation, data cleaning
    \item Classification - including diagnosis and \textbf{Anomaly Detection}
    \item Reinforcement Learning
\end{itemize}

And then for the future, how does it relate to \textbf{Causal Learning}?

\subsection{Abstraction}
\subsection{Regression}
\subsection{Classifcation}
\subsection{Reinforcement Learning}
what was I going to say? \bla{soo much RL happening!}

In the past year and a half, the lab has published a \bla{stunning} range of solutions to very challenging RL problems. 
We approach RL from a different direction from many researchers, we look for edge cases and connections to real world decision making that standard RL approaches have trouble handling.

At the moment there are \bla{three} thrusts to our RL research:
\begin{itemize}
    \item \bla{Scientist Agents} - domains or \bla{setups} that relate to the situation of scientist as agent. 
    \item Multi-Agent RL - \bla{this is a growing area for us with many things going on.}
    \item what else?
\end{itemize}

\subsubsection{Scientist Agents}
Although outside the RL approach, we also presented work in late 2019 in Germany at ECML describing our method for \bla{predicting combustion apramters} \cite{bhalla2019ecml}.

The core idea here to discover how to \bla{learn like a scientists, actions can be observations, and they can have different costs.}
This work is funded by a collaboration between the University of Waterloo (UW) and the National Research Council (NRC) on a project for \emph{Material Design and Discovery using Reinforcement Learning}.
We've published two papers on this topic now \cite{bellinger2020reinforcement} and \cite{bellinger2021canai} on different aspects of the problem.


\subsubsection{MARL}
Communication papers \cite{Bhalla2019aamas} and the MASc thesis of Sushrut Bhalla \todo{sushrut thesis link}.
Stackleberg approach to autonomous driving and the results of MASc thesis of Sahil Pereira \todo{sahil thesis link}.




\subsection{Causal Learning}

In the are of \textit{Anomaly Detection} we finally published a nice little paper on an idea I've long wanted to put out there, \citep*{ganapathisubramanian2021aamas} 


The amazing results of \citep*{bhalla2019ecml} which I presented the previous year... oh there's already a blog for that

One of our first AAMAS papers was the core MASc thesis work of Sushrut Bhalla \citep*{Bhalla2019aamas} on the problem of \textit{learning communication protocol} in Multi-Agent Reinforcement Learning.

\section{By Research Project/Topic}

\todo{from slate/Tenure-Planning, edit and make them more clear}

\subsection{Domain: Medical Imaging}
My research into medical imaging has two main target domains, \textbf{Alzheimer's Disease} and \textbf{Digital Pathology}. For Alzheimer's, we look at prediction and diagnosis using \textit{Diffusion MRI} brain scan data. While for Digital Pathology we look at a range of classification, diagnosis and search tasks to enable the effective usage of this high resolution image and associated 
natural language medical report data. The reports are written by expert pathologists diagnosing their patients based on high resolution, digital microscope images of biopsied tissue samples. The challenge of this domain is the extremely large size of the images (over 4000x4000 pixels) and the irregular nature of the medical reports. 

I am a co-investigator on a \todo{large CRF grant} on the topic of Digital Pathology \todo{(see \url{kimialab} for more information on this project)}. We have published 6 papers on this topic with another 1 currently under review and several more in the pipeline.
My long term goal here is to push the boundary of deep learning methods used in this domain by discovering how to efficiently and effectively fuse data from multiple, very different, sources of image
and text. 

\bla{This will also, later on, feed into my next research focus on causal model learning , which will span many projects in the lab.}

\subsection{Causality}

A unifying theme across all the domains I work on is the presence of well known, but poorly understood, causal \bla{relationships} in the system dynamics. All of these domains change in ways that are the result of the interaction of many variables in a way that is modelled perfectly and which may also be inherently stochastic.

In all these domains then, it would be beneficial to be able to reason about whether certain variables have a causal relationship or merely correlational one.

\subsection{Autonomous Driving}

A car itself can be highly engineered and treated as a dynamic 
control problem under ideal conditions.
However, this would never not be sufficient to drive a car fully autonomously in a world where everything around the vehicle is controlled by stochastic and unpredictable people or natural processes.

Human drivers take into account vast amounts of data about their surroundings as well as a trained understanding of how the vehicle will respond to various controls. 
But these two complex achievements alone are still not enough to safely drive a car in real conditions. 
The driver also needs to have a predictive, causal model of ``how the world works'' from the point of view of a vehicle on the road. 
This includes the physics of objects at high speed, the effects of friction, ice, rain, and even wind.
I argue, it must also include causal models of human behaviour for pedestrians and other vehicles. 
The driver \textit{knows} that a car heading straight towards them but with a turn signal activated will almost certainly slow down and wait or turn away from them before collision.
Unless we embed that knowledge somewhere, or discover how to let agents \textit{learn such reasoning abilities}, then autonomous driving systems will never be able to compare to human performance.
These are very complex causal relations that are present in the data, and add be learned. 
Initial, heuristic rules capturing normal situation can, and should , be integrated into deployed systems. 
But these rules cannot hope to capture all possible, relevant causal relations. 
So. learning should be part of the system as well and since the task of driving is quite universal, data from many drivers can be used merged together.


\newpage
\section{Paper Lists}
\subsection{Published 2021}
\subsubsection[CANAI-2021 (bellinger2021canai)]{\textbf{CanAI-2021} : Active Measure Reinforcement Learning for Observation Cost Minimization: A framework for minimizing measurement costs in reinforcement learning}
%bellinger2021canai
%status: 1
\begin{itemize}
\item \textbf{keywords:} reinforcement-learning, Machine-Learning, chemistry, proj-deepchemrl, showcase
\item \textbf{Arxiv:} 2005.12697 (if any)
\end{itemize}
%todo: type (conference, journal, ptatne...)
%todo: can we check if a field is empy?
%todo: copy all paperdesc text from cv to bibtex in this way
%todo: create another version of this template for markdown, with a page separator

\fbox{%
\parbox{\textwidth}{%
        \textbf{CITATION: \texttt{bellinger2021canai} \cite{bellinger2021canai}}:\\
            Colin Bellinger, Rory Coles, Mark Crowley, and Isaac Tamblyn.  ``Active Measure Reinforcement Learning for Observation Cost Minimization: A framework for minimizing measurement costs in reinforcement learning'', \emph{}. In \emph{Canadian Conference on Artificial Intelligence}. (), 12 pages. , 2021.
    }%
}

\textbf{Abstract:} Markov Decision Processes (MDP) with explicit measurement cost are a class of en- vironments in which the agent learns to maximize the costed return. Here, we define the costed return as the discounted sum of rewards minus the sum of the explicit cost of measuring the next state. The RL agent can freely explore the relationship between actions and rewards but is charged each time it measures the next state. Thus, an op- timal agent must learn a policy without making a large number of measurements. We propose the active measure RL framework (Amrl) as a solution to this novel class of problem, and contrast it with standard reinforcement learning under full observability and planning under partially observability. We demonstrate that Amrl-Q agents learn to shift from a reliance on costly measurements to exploiting a learned transition model in order to reduce the number of real-world measurements and achieve a higher costed return. Our results demonstrate the superiority of Amrl-Q over standard RL methods, Q-learning and Dyna-Q, and POMCP for planning under a POMDP in environments with explicit measurement costs.

\textbf{paperdesc:} Bellinger and Tamblyn are researchers at NRC. They carried out some experiments based on my idea of extending a standard RL approach to include observation costs, whcih is necessary to enable automated control of scientific experiments. This work arises from the NRC-UW collaboration grant I lead. This builds on work from \cite{bellinger2020reinforcement} last year.

\textbf{Description:} 



\newpage
\subsubsection{\textbf{AAMAS-2021} : Partially Observable Mean Field Reinforcement Learning}
%ganapathisubramanian2021aamas
%status: 1
\begin{itemize}
\item \textbf{keywords:} marl, reinforcement-learning, mean field theory, partial observation, showcase
\item \textbf{Arxiv:} 2012.15791 (if any)
\end{itemize}
%todo: type (conference, journal, ptatne...)
%todo: can we check if a field is empy?
%todo: copy all paperdesc text from cv to bibtex in this way
%todo: create another version of this template for markdown, with a page separator

\fbox{%
\parbox{\textwidth}{%
        \textbf{CITATION: \texttt{ganapathisubramanian2021aamas} \cite{ganapathisubramanian2021aamas}}:\\
            Sriram Ganapathi Subramanian, Matthew Taylor, Mark Crowley, and Pascal Poupart.  ``Partially Observable Mean Field Reinforcement Learning'', \emph{}. In \emph{Proceedings of the 20th International Conference on Autonomous Agents and MultiAgent Systems (AAMAS)}. (), 537-545 pages. London, United Kingdom, 2021.
    }%
}

\textbf{Abstract:} Traditional multi-agent reinforcement learning algorithms are not scalable to environments with more than a few agents, since these algorithms are exponential in the number of agents. Recent research has introduced successful methods to scale multi-agent reinforcement learning algorithms to many agent scenarios using mean field theory. Previous work in this field assumes that an agent has access to exact cumulative metrics regarding the mean field behaviour of the system, which it can then use to take its actions. In this paper, we relax this assumption and maintain a distribution to model the uncertainty regarding the mean field of the system. We consider two different settings for this problem. In the first setting, only agents in a fixed neighbourhood are visible, while in the second setting, the visibility of agents is determined at random based on distances. For each of these settings, we introduce a Q-learning based algorithm that can learn effectively. We prove that this Q-learning estimate stays very close to the Nash Q-value (under a common set of assumptions) for the first setting. We also empirically show our algorithms outperform multiple baselines in three different games in the MAgents framework, which supports large environments with many agents learning simultaneously to achieve possibly distinct goals.

\textbf{paperdesc:} This paper submission extends work which my PhD student Ganapathi Subramanian has done previously while hired as an industry co-op term with Poupart. The core idea came from the student himself. The writing and theory was a full collaboration between all contributors through weekly calls and collaborative writing.

\textbf{Description:} 



\newpage
\subsubsection{\textbf{ISBI-2021} : Magnification Generalization for Histopathology Image Embedding}
%sikaroudi2021isbi
%status: 1
\begin{itemize}
\item \textbf{keywords:} 
\item \textbf{Arxiv:} 2101.07757 (if any)
\end{itemize}
%todo: type (conference, journal, ptatne...)
%todo: can we check if a field is empy?
%todo: copy all paperdesc text from cv to bibtex in this way
%todo: create another version of this template for markdown, with a page separator

\fbox{%
\parbox{\textwidth}{%
        \textbf{CITATION: \texttt{sikaroudi2021isbi} \cite{sikaroudi2021isbi}}:\\
            Milad Sikaroudi, Benyamin Ghojogh, Fakhri Karray, Mark Crowley, and H. R. Tizhoosh.  ``Magnification Generalization for Histopathology Image Embedding'', \emph{}. In \emph{IEEE International Symposium on Biomedical Imaging (ISBI)}. (), 5 pages. , 2021.
    }%
}

\textbf{Abstract:} Histopathology image embedding is an active research area in computer vision. Most of the embedding models exclusively concentrate on a specific magnification level. However, a useful task in histopathology embedding is to train an embedding space regardless of the magnification level. Two main approaches for tackling this goal are domain adaptation and domain generalization, where the target magnification levels may or may not be introduced to the model in training, respectively. Although magnification adaptation is a well-studied topic in the literature, this paper, to the best of our knowledge, is the first work on magnification generalization for histopathology image embedding. We use an episodic trainable domain generalization technique for magnification generalization, namely Model Agnostic Learning of Semantic Features (MASF), which works based on the Model Agnostic Meta-Learning (MAML) concept. Our experimental results on a breast cancer histopathology dataset with four different magnification levels show the proposed method's effectiveness for magnification generalization.

\textbf{paperdesc:} 

\textbf{Description:} 



\newpage
\subsubsection{\textbf{ICPR-2021} : Batch-Incremental Triplet Sampling for Training Triplet Networks Using Bayesian Updating Theorem}
%sikaroudi2021icpr
%status: 1
\begin{itemize}
\item \textbf{keywords:} showcase
\item \textbf{Arxiv:} 2007.05610 (if any)
\end{itemize}
%todo: type (conference, journal, ptatne...)
%todo: can we check if a field is empy?
%todo: copy all paperdesc text from cv to bibtex in this way
%todo: create another version of this template for markdown, with a page separator

\fbox{%
\parbox{\textwidth}{%
        \textbf{CITATION: \texttt{sikaroudi2021icpr} \cite{sikaroudi2021icpr}}:\\
            Milad Sikaroudi, Benyamin Ghojogh, Fakhri Karray, Mark Crowley, and H. R. Tizhoosh.  ``Batch-Incremental Triplet Sampling for Training Triplet Networks Using Bayesian Updating Theorem'', \emph{}. In \emph{25th International Conference on Pattern Recognition (ICPR)}. (), 7 pages. Milan, Italy (virtual), 2021.
    }%
}

\textbf{Abstract:} Variants of Triplet networks are robust entities for learning a discriminative embedding subspace. There exist different triplet mining approaches for selecting the most suitable training triplets. Some of these mining methods rely on the extreme distances between instances, and some others make use of sampling. However, sampling from stochastic distributions of data rather than sampling merely from the existing embedding instances can provide more discriminative information. In this work, we sample triplets from distributions of data rather than from existing instances. We consider a multivariate normal distribution for the embedding of each class. Using Bayesian updating and conjugate priors, we update the distributions of classes dynamically by receiving the new mini-batches of training data. The proposed triplet mining with Bayesian updating can be used with any triplet-based loss function, e.g., triplet-loss or Neighborhood Component Analysis (NCA) loss. Accordingly, Our triplet mining approaches are called Bayesian Updating Triplet (BUT) and Bayesian Updating NCA (BUNCA), depending on which loss function is being used. Experimental results on two public datasets, namely MNIST and histopathology colorectal cancer (CRC), substantiate the effectiveness of the proposed triplet mining method.

\textbf{paperdesc:} 

\textbf{Description:} 







\newpage
\subsection{Published 2020}
\subsubsection{\textbf{CanAI-2020} : Reinforcement Learning in a Physics-Inspired Semi-Markov Environment}
%bellinger2020reinforcement
%status: 1
\begin{itemize}
\item \textbf{keywords:} Reinforcement-Learning, proj-deepchemrl, deep learning, showcase
\item \textbf{Arxiv:}  (if any)
\end{itemize}
%todo: type (conference, journal, ptatne...)
%todo: can we check if a field is empy?
%todo: copy all paperdesc text from cv to bibtex in this way
%todo: create another version of this template for markdown, with a page separator

\fbox{%
\parbox{\textwidth}{%
        \textbf{CITATION: \texttt{bellinger2020reinforcement} \cite{bellinger2020reinforcement}}:\\
            Colin Bellinger, Rory Coles, Mark Crowley, and Isaac Tamblyn.  ``Reinforcement Learning in a Physics-Inspired Semi-Markov Environment'', \emph{}. In \emph{Canadian Conference on Artificial Intelligence}. 12109(), 55-66 pages. Ottawa, Canada (virtual), 2020.
    }%
}

\textbf{Abstract:} 

\textbf{paperdesc:} I was intimately involved in creation of the core idea of this paper and was involved in writing, development and multiple revisions. It is resulting from a post doctoral fellow (Bellinger) working under Tamblyn at NRC on our joint project using Reinforcement Learning for Material Design and Discovery.

\textbf{Description:} 



\newpage
\subsubsection{\textbf{EMBC-2020} : Supervision and Source Domain Impact on Representation Learning: A Histopathology Case Study}
%sikaroudi2020embc
%status: 1
\begin{itemize}
\item \textbf{keywords:} representation-learning, proj-digipath, proj-digipath, proj-digipath, medical, showcase
\item \textbf{Arxiv:}  (if any)
\end{itemize}
%todo: type (conference, journal, ptatne...)
%todo: can we check if a field is empy?
%todo: copy all paperdesc text from cv to bibtex in this way
%todo: create another version of this template for markdown, with a page separator

\fbox{%
\parbox{\textwidth}{%
        \textbf{CITATION: \texttt{sikaroudi2020embc} \cite{sikaroudi2020embc}}:\\
            Milad Sikaroudi, Amir Safarpoor, Benyamin Ghojogh, Sobhan Shafiei, Mark Crowley, and HR Tizhoosh.  ``Supervision and Source Domain Impact on Representation Learning: A Histopathology Case Study'', \emph{}. In \emph{International Conference of the IEEE Engineering in Medicine and Biology Society (EMBC'20)}. (), 4 pages. Montreal, Quebec, Canada (virtual), 2020.
    }%
}

\textbf{Abstract:} As many algorithms depend on a suitable representation of data, learning
unique features is considered a crucial task. Although supervised techniques
using deep neural networks have boosted the performance of representation
learning, the need for a large set of labeled data limits the application of
such methods. As an example, high-quality delineations of regions of interest
in the field of pathology is a tedious and time-consuming task due to the large
image dimensions. In this work, we explored the performance of a deep neural
network and triplet loss in the area of representation learning. We
investigated the notion of similarity and dissimilarity in pathology
whole-slide images and compared different setups from unsupervised and
semi-supervised to supervised learning in our experiments. Additionally,
different approaches were tested, applying few-shot learning on two publicly
available pathology image datasets. We achieved high accuracy and
generalization when the learned representations were applied to two different
pathology datasets

\textbf{paperdesc:} IJCNN is very well respected conference on neural networks held as part of the IEEE World Congress on Computational Intelligence (WCCI). This paper expands on the new trend of methods utilizing variable triplets to build "Siamese" deep nerural networks, in this paper we provide an improved method for training such models which is particularly useful for complex image understanding.

\textbf{Description:} 



\newpage
\subsubsection{\textbf{CDC-2020} : Distributed Nonlinear Model Predictive Control and Metric Learning for Heterogeneous Vehicle Platooning with Cut-in/Cut-out Maneuvers}
%basiri2020cdc
%status: 1
\begin{itemize}
\item \textbf{keywords:} showcase
\item \textbf{Arxiv:}  (if any)
\end{itemize}
%todo: type (conference, journal, ptatne...)
%todo: can we check if a field is empy?
%todo: copy all paperdesc text from cv to bibtex in this way
%todo: create another version of this template for markdown, with a page separator

\fbox{%
\parbox{\textwidth}{%
        \textbf{CITATION: \texttt{basiri2020cdc} \cite{basiri2020cdc}}:\\
            Mohammad Hossein Basiri, Benyamin Ghojogh, Nasser L Azad, Sebastian Fischmeister, Fakhri Karray, and Mark Crowley.  ``Distributed Nonlinear Model Predictive Control and Metric Learning for Heterogeneous Vehicle Platooning with Cut-in/Cut-out Maneuvers'', \emph{arXiv preprint arXiv:2004.00417}. In \emph{Proceeding of the 59th IEEE Conference on Decision and Control (CDC-2020)}. (), 2849-2856 pages. Jeju Island, Korea (virtual), 2020.
    }%
}

\textbf{Abstract:} 

\textbf{paperdesc:} This collaboration paper deals with platoon driving, combining use of control algorithms with metric learning. I worked with my student Ghojogh on the theoretical metric learning component and was involved heavily in paper editing and revision.

\textbf{Description:} 



\newpage
\subsubsection{\textbf{ICIAR-2020} : Generalized Subspace Learning by Roweis Discriminant Analysis}
%ghojogh2019rda
%status: 1
\begin{itemize}
\item \textbf{keywords:} Manifold-Learning, Data Reduction, Numerosity Reduction
\item \textbf{Arxiv:}  (if any)
\end{itemize}
%todo: type (conference, journal, ptatne...)
%todo: can we check if a field is empy?
%todo: copy all paperdesc text from cv to bibtex in this way
%todo: create another version of this template for markdown, with a page separator

\fbox{%
\parbox{\textwidth}{%
        \textbf{CITATION: \texttt{ghojogh2019rda} \cite{ghojogh2019rda}}:\\
            Benyamin Ghojogh, Fakhri Karray, and Mark Crowley.  ``Generalized Subspace Learning by Roweis Discriminant Analysis'', \emph{}. In \emph{International Conference on Image Analysis and Recognition (ICIAR-2020)}. (), 9 pages. Póvoa de Varzim, Portugal (virtual), 2020.
    }%
}

\textbf{Abstract:} We present a new method which generalizes subspace learning based on eigenvalue and generalized eigenvalue problems. This method, Roweis Discriminant Analysis (RDA), is named after Sam Roweis to whom the field of subspace learning owes significantly. RDA is a family of infinite number of algorithms where Principal Component Analysis (PCA), Supervised PCA (SPCA), and Fisher Discriminant Analysis (FDA) are special cases. One of the extreme special cases, which we name Double Supervised Discriminant Analysis (DSDA), uses the labels twice, it is novel and has not appeared elsewhere. We propose a dual for RDA for some special cases. We also propose kernel RDA, generalizing kernel PCA, kernel SPCA, and kernel FDA, using both dual RDA and representation theory. Our theoretical analysis explains previously known facts such as why SPCA can use regression but FDA cannot, why PCA and SPCA have duals but FDA does not, why kernel PCA and kernel SPCA use kernel trick but kernel FDA does not, and why PCA is the best linear method for reconstruction. Roweisfaces and kernel Roweisfaces are also proposed generalizing eigenfaces, Fisherfaces, supervised eigenfaces, and their kernel variants. We also report experiments showing the effectiveness of RDA and kernel RDA on some benchmark datasets.

\textbf{paperdesc:} This is a core theoretical contribution towards the thesis of my PhD student Ghojogh. We discussed the central idea of this work at length, refining the breadth of the theory and how to explain it. He carried out extensive experiement and writing drafts.

\textbf{Description:} 



\newpage
\subsubsection{\textbf{ICMI-2020} : Using Emotions to Complement Multi-Modal Human-Robot Interaction in Urban Search and Rescue Scenarios}
%akgun2020icmi
%status: 1
\begin{itemize}
\item \textbf{keywords:} search-and-rescue, urban search and rescue, multi-modal communication, affective communication, emotions, human-robot interaction
\item \textbf{Arxiv:}  (if any)
\end{itemize}
%todo: type (conference, journal, ptatne...)
%todo: can we check if a field is empy?
%todo: copy all paperdesc text from cv to bibtex in this way
%todo: create another version of this template for markdown, with a page separator

\fbox{%
\parbox{\textwidth}{%
        \textbf{CITATION: \texttt{akgun2020icmi} \cite{akgun2020icmi}}:\\
            Sami Alperen Akgun, Moojan Ghafurian, Mark Crowley, and Kerstin Dautenhahn.  ``Using Emotions to Complement Multi-Modal Human-Robot Interaction in Urban Search and Rescue Scenarios'', \emph{}. In \emph{Proceedings of the 2020 International Conference on Multimodal Interaction (ICMI)}. (), 575--584 pages. Utrecht, the Netherlands (virtual), 2020.
    }%
}

\textbf{Abstract:} An experiment is presented to investigate whether there is consensus in mapping emotions to messages/situations in urban search and rescue scenarios, where efficiency and effectiveness of interactions are key to success. We studied mappings between 10 specific messages, presented in two different communication styles, reflecting common situations that might happen during search and rescue missions, and the emotions exhibited by robots in those situations. The data was obtained through a Mechanical Turk study with 78 participants. Our findings support the feasibility of using emotions as an additional communication channel to improve multi-modal human-robot interaction for urban search and rescue robots, and suggests that these mappings are robust, i.e. are not affected by the robot's communication style.

\textbf{paperdesc:} This is a highly competitive conference in the field of complex, multi-model robotic systems and human interaction. Our paper is a study of human interpretation of robot actions intended to mimic emotions and how they can be used to increase human understanding of robot responses in the presence of visual distractions as would occur during disaster recovery operations. I consulted heavily on the analysis approaches being used and connections to future RL use cases.

\textbf{Description:} 



\newpage
\subsubsection{\textbf{SMC-2020} : Isolation Mondrian Forest for Batch and Online Anomaly Detection}
%ma2020imondrian
%status: 1
\begin{itemize}
\item \textbf{keywords:} showcase
\item \textbf{Arxiv:}  (if any)
\end{itemize}
%todo: type (conference, journal, ptatne...)
%todo: can we check if a field is empy?
%todo: copy all paperdesc text from cv to bibtex in this way
%todo: create another version of this template for markdown, with a page separator

\fbox{%
\parbox{\textwidth}{%
        \textbf{CITATION: \texttt{ma2020imondrian} \cite{ma2020imondrian}}:\\
            Haoran Ma, Benyamin Ghojogh, Maria N Samad, Dongyu Zheng, and Mark Crowley.  ``Isolation Mondrian Forest for Batch and Online Anomaly Detection'', \emph{}. In \emph{IEEE International Conference on Systems, Man, and Cybernetics (IEEE-SMC-2020)}. (), 7 pages. Toronto, Canada (virtual), 2020.
    }%
}

\textbf{Abstract:} 

\textbf{paperdesc:} This is a novel anomaly detection algorithm which I came up with and put forward to graduate student Samad and was then fleshed out and expanded with the other students. I was centrally involved in all aspects of this paper, including writing and I also presented it at the virtual conference. The algorithm fuses two ideas, "isolation" from ensemble trees methods for anomaly detection and "Mondrian forests" which can learn flexible regression models from streaming data.

\textbf{Description:} This is a novel anomaly detection algorithm which I came up with and put forward to graduate student Samad and was then fleshed out and expanded with the other students. I was centrally involved in all aspects of this paper, including writing and I also presented it at the virtual conference. 
    The algorithm fuses two ideas, "isolation" from ensemble trees methods for anomaly detection and "Mondrian forests" which can learn flexible regression models from streaming data.



\newpage
\subsubsection{\textbf{EnvRevJrnl-2020} : A review of machine learning applications in wildfire science and management}
%jain2020review
%status: 1
\begin{itemize}
\item \textbf{keywords:} showcase
\item \textbf{Arxiv:} 2003.00646 (if any)
\end{itemize}
%todo: type (conference, journal, ptatne...)
%todo: can we check if a field is empy?
%todo: copy all paperdesc text from cv to bibtex in this way
%todo: create another version of this template for markdown, with a page separator

\fbox{%
\parbox{\textwidth}{%
        \textbf{CITATION: \texttt{jain2020review} \cite{jain2020review}}:\\
            Piyush Jain, Sean CP Coogan, Sriram Ganapathi Subramanian, Mark Crowley, Steve Taylor, and Mike D Flannigan.  ``A review of machine learning applications in wildfire science and management'', \emph{Environmental Reviews}. In \emph{}. 28(3),  pages. , 2020.
    }%
}

\textbf{Abstract:} 

\textbf{paperdesc:} This review paper arose from my visit to BC to speak on the use of AI for Forest Fire management and led to a collaboration amongst these senior researchers, myself and my PhD student Sriram. We collaborated on every part of the paper, I especially wrote the general AI/ML background and checked that each specific Forest Fire domain was connected correctly to the ML literature. It will serve as a much-needed resource for researches in my field as well as applied Forest Fire Management and Science fields.

\textbf{Description:} 



\newpage
\subsubsection{\textbf{ICMI-2020} : Recognition of a Robot's Affective Expressions under Conditions with Limited Visibility}
%ghafurian2020icmi
%status: 1
\begin{itemize}
\item \textbf{keywords:} showcase, human-robot interaction
\item \textbf{Arxiv:}  (if any)
\end{itemize}
%todo: type (conference, journal, ptatne...)
%todo: can we check if a field is empy?
%todo: copy all paperdesc text from cv to bibtex in this way
%todo: create another version of this template for markdown, with a page separator

\fbox{%
\parbox{\textwidth}{%
        \textbf{CITATION: \texttt{ghafurian2020icmi} \cite{ghafurian2020icmi}}:\\
            Moojan Ghafurian, Sami Alperen Akgun, Mark Crowley, and Kerstin Dautenhahn.  ``Recognition of a Robot's Affective Expressions under Conditions with Limited Visibility'', \emph{}. In \emph{International Conference on Human-Robot Interaction}. (),  pages. Cambridge, UK (virtual), 2020.
    }%
}

\textbf{Abstract:} The capability of showing affective expressions is important for the design of social robots in many contexts, especially where the robot is designed to communicate with humans. It is reasonable to expect that, similar to all other interaction modalities, communicating with affective expressions is not without limitations. In this paper, we present two online video studies (N=72 and N=50) and investigate if/how much the recognition of affective displays of a zoomorphic robot is affected under situations with different levels of visibility. Five affective expressions combined with five visibility effects were studied. The intensity of the effects was more pronounced in the second experiment. While visual constraints affected recognition of expressions, our results showed that affective displays of the robot conveyed through its head and body gestures can be robust and recognition rates can still be high even under severe constraints. This study supported the effectiveness of using affective displays as a complementary communication modality in human-robot interaction in situations with low visibility.

\textbf{paperdesc:} 

\textbf{Description:} 



\newpage
\subsubsection{\textbf{ISCV-2020} : Offline versus Online Triplet Mining based on Extreme Distances of Histopathology Patches}
%sikaroudi2020iscv
%status: 1
\begin{itemize}
\item \textbf{keywords:} proj-digipath, triplet mining, machine learning, proj-digipath
\item \textbf{Arxiv:} 2007.02200 (if any)
\end{itemize}
%todo: type (conference, journal, ptatne...)
%todo: can we check if a field is empy?
%todo: copy all paperdesc text from cv to bibtex in this way
%todo: create another version of this template for markdown, with a page separator

\fbox{%
\parbox{\textwidth}{%
        \textbf{CITATION: \texttt{sikaroudi2020iscv} \cite{sikaroudi2020iscv}}:\\
            Milad Sikaroudi, Benyamin Ghojogh, Amir Safarpoor, Fakhri Karray, Mark Crowley, and H. R. Tizhoosh.  ``Offline versus Online Triplet Mining based on Extreme Distances of Histopathology Patches'', \emph{}. In \emph{15th International Symposium on Visual Computing (ISCV 2020)}. (), 333--345 pages. (virtual), 2020.
    }%
}

\textbf{Abstract:} We analyze the effect of offline and online triplet mining for colorectal cancer (CRC) histopathology dataset containing 100,000 patches. We consider the extreme, i.e., farthest and nearest patches with respect to a given anchor, both in online and offline mining. While many works focus solely on how to select the triplets online (batch-wise), we also study the effect of extreme distances and neighbor patches before training in an offline fashion. We analyze the impacts of extreme cases for offline versus online mining, including easy positive, batch semi-hard, and batch hard triplet mining as well as the neighborhood component analysis loss, its proxy version, and distance weighted sampling. We also investigate online approaches based on extreme distance and comprehensively compare the performance of offline and online mining based on the data patterns and explain offline mining as a tractable generalization of the online mining with large mini-batch size. As well, we discuss the relations of different colorectal tissue types in terms of extreme distances. We found that offline mining can generate a better statistical representation of the population by working on the whole dataset.

\textbf{paperdesc:} 

\textbf{Description:} 



\newpage
\subsubsection{\textbf{Patent-2020} : Multi-Level Collaborative Control System With Dual Neural Network Planning For Autonomous Vehicle Control In A Noisy Environment}
%patent
%status: 1
\begin{itemize}
\item \textbf{keywords:} patentpending, DENSO, autonomous-driving
\item \textbf{Arxiv:}  (if any)
\end{itemize}
%todo: type (conference, journal, ptatne...)
%todo: can we check if a field is empy?
%todo: copy all paperdesc text from cv to bibtex in this way
%todo: create another version of this template for markdown, with a page separator

\fbox{%
\parbox{\textwidth}{%
        \textbf{CITATION: \texttt{patent} \cite{patent}}:\\
            Zhiyuan Du, Joseph Lull, Rajesh Malhan, Sriram Ganapathi Subramanian, Sushrut Bhalla, Jaspreet Sambee, and Mark Crowley.  ``Multi-Level Collaborative Control System With Dual Neural Network Planning For Autonomous Vehicle Control In A Noisy Environment'', \emph{}. In \emph{}. (),  pages. , 2020.
    }%
}

\textbf{Abstract:} 

\textbf{paperdesc:} 

\textbf{Description:} 



\newpage
\subsubsection{\textbf{CanAI-2020} : Deep Multi Agent Reinforcement Learning for Autonomous Driving}
%bhalla2020deep
%status: 1
\begin{itemize}
\item \textbf{keywords:} showcase, MARL
\item \textbf{Arxiv:}  (if any)
\end{itemize}
%todo: type (conference, journal, ptatne...)
%todo: can we check if a field is empy?
%todo: copy all paperdesc text from cv to bibtex in this way
%todo: create another version of this template for markdown, with a page separator

\fbox{%
\parbox{\textwidth}{%
        \textbf{CITATION: \texttt{bhalla2020deep} \cite{bhalla2020deep}}:\\
            Sushrut Bhalla, Sriram Ganapathi Subramanian, and Mark Crowley.  ``Deep Multi Agent Reinforcement Learning for Autonomous Driving'', \emph{}. In \emph{Canadian Conference on Artificial Intelligence}. (), 67--78 pages. , 2020.
    }%
}

\textbf{Abstract:} 

\textbf{paperdesc:} 

\textbf{Description:} 



\newpage
\subsubsection{\textbf{IJCNN-2020} : Fisher Discriminant Triplet and Contrastive Losses for Training Siamese Networks}
%ghojogh2020fisher
%status: 1
\begin{itemize}
\item \textbf{keywords:} 
\item \textbf{Arxiv:}  (if any)
\end{itemize}
%todo: type (conference, journal, ptatne...)
%todo: can we check if a field is empy?
%todo: copy all paperdesc text from cv to bibtex in this way
%todo: create another version of this template for markdown, with a page separator


\fbox{%
\parbox{\textwidth}{%
        \textbf{CITATION: \texttt{ghojogh2020fisher} \cite{ghojogh2020fisher}}:\\
            Benyamin Ghojogh, Milad Sikaroudi, Sobhan Shafiei, H.R. Tizhoosh, Fakhri Karray, and Mark Crowley.  ``Fisher Discriminant Triplet and Contrastive Losses for Training Siamese Networks'', \emph{}. In \emph{IEEE International Joint Conference on Neural Networks (IJCNN)}. (),  pages. Glasgow, UK (virtual), 2020.
    }%
}

\textbf{Abstract:} 

\textbf{paperdesc:} 

\textbf{Description:} 



\newpage
\subsubsection{\textbf{ICIAR-2020} : Weighted Fisher Discriminant Analysis in the Input and Feature Spaces}
%ghojogh2020weighted
%status: 1
\begin{itemize}
\item \textbf{keywords:} Manifold-Learning
\item \textbf{Arxiv:}  (if any)
\end{itemize}
%todo: type (conference, journal, ptatne...)
%todo: can we check if a field is empy?
%todo: copy all paperdesc text from cv to bibtex in this way
%todo: create another version of this template for markdown, with a page separator

\fbox{%
\parbox{\textwidth}{%
        \textbf{CITATION: \texttt{ghojogh2020weighted} \cite{ghojogh2020weighted}}:\\
            Benyamin Ghojogh, Milad Sikaroudi, H.R. Tizhoosh, Fakhri Karray, and Mark Crowley.  ``Weighted Fisher Discriminant Analysis in the Input and Feature Spaces'', \emph{}. In \emph{International Conference on Image Analysis and Recognition (ICIAR-2020)}. (), 9 pages. Póvoa de Varzim, Portugal (virtual), 2020.
    }%
}

\textbf{Abstract:} 

\textbf{paperdesc:} 

\textbf{Description:} 



\newpage
\subsubsection{\textbf{ICIAR-2020} : Backprojection for Training Feedforward Neural Networks in the Input and Feature Spaces}
%ghojogh2020backprojection
%status: 1
\begin{itemize}
\item \textbf{keywords:} Deep Neural Networks, neural networks
\item \textbf{Arxiv:}  (if any)
\end{itemize}
%todo: type (conference, journal, ptatne...)
%todo: can we check if a field is empy?
%todo: copy all paperdesc text from cv to bibtex in this way
%todo: create another version of this template for markdown, with a page separator

\fbox{%
\parbox{\textwidth}{%
        \textbf{CITATION: \texttt{ghojogh2020backprojection} \cite{ghojogh2020backprojection}}:\\
            Benyamin Ghojogh, Fakhri Karray, and Mark Crowley.  ``Backprojection for Training Feedforward Neural Networks in the Input and Feature Spaces'', \emph{}. In \emph{International Conference on Image Analysis and Recognition (ICIAR-2020)}. (), 9 pages. Póvoa de Varzim, Portugal (virtual), 2020.
    }%
}

\textbf{Abstract:} 

\textbf{paperdesc:} 

\textbf{Description:} 



\newpage
\subsubsection{\textbf{ICIAR-2020} : Theoretical Insights into the Use of Structural Similarity Index In Generative Models and Inferential Autoencoders}
%ghojogh2020theoretical
%status: 1
\begin{itemize}
\item \textbf{keywords:} image understanding, Manifold-Learning, representation-learning
\item \textbf{Arxiv:}  (if any)
\end{itemize}
%todo: type (conference, journal, ptatne...)
%todo: can we check if a field is empy?
%todo: copy all paperdesc text from cv to bibtex in this way
%todo: create another version of this template for markdown, with a page separator

\fbox{%
\parbox{\textwidth}{%
        \textbf{CITATION: \texttt{ghojogh2020theoretical} \cite{ghojogh2020theoretical}}:\\
            Benyamin Ghojogh, Fakhri Karray, and Mark Crowley.  ``Theoretical Insights into the Use of Structural Similarity Index In Generative Models and Inferential Autoencoders'', \emph{}. In \emph{International Conference on Image Analysis and Recognition (ICIAR-2020)}. (), 9 pages. Póvoa de Varzim, Portugal (virtual), 2020.
    }%
}

\textbf{Abstract:} 

\textbf{paperdesc:} 

\textbf{Description:} 



\newpage
\subsubsection{\textbf{CanAI-2020} : Anomaly Detection and Prototype Selection Using Polyhedron Curvature}
%ghojogh2020anomaly
%status: 1
\begin{itemize}
\item \textbf{keywords:} anomaly detection, representation-learning
\item \textbf{Arxiv:}  (if any)
\end{itemize}
%todo: type (conference, journal, ptatne...)
%todo: can we check if a field is empy?
%todo: copy all paperdesc text from cv to bibtex in this way
%todo: create another version of this template for markdown, with a page separator

\fbox{%
\parbox{\textwidth}{%
        \textbf{CITATION: \texttt{ghojogh2020anomaly} \cite{ghojogh2020anomaly}}:\\
            Benyamin Ghojogh, Fakhri Karray, and Mark Crowley.  ``Anomaly Detection and Prototype Selection Using Polyhedron Curvature'', \emph{}. In \emph{Canadian Conference on Artificial Intelligence}. (), 10 pages. Ottawa, Canada, 2020.
    }%
}

\textbf{Abstract:} 

\textbf{paperdesc:} 

\textbf{Description:} 







\newpage
\subsection{Published 2019}
\subsubsection{\textbf{AAMAS-2019} : Training Cooperative Agents for Multi-Agent Reinforcement Learning}
%Bhalla2019aamas
%status: 1
\begin{itemize}
\item \textbf{keywords:} Autonomous Driving,MARL,MARL,Multi-Agent-Systems,Reinforcement-Learning
\item \textbf{Arxiv:}  (if any)
\end{itemize}
%todo: type (conference, journal, ptatne...)
%todo: can we check if a field is empy?
%todo: copy all paperdesc text from cv to bibtex in this way
%todo: create another version of this template for markdown, with a page separator

\fbox{%
\parbox{\textwidth}{%
        \textbf{CITATION: \texttt{Bhalla2019aamas} \cite{Bhalla2019aamas}}:\\
            Sushrut Bhalla, Sriram Ganapathi Subramanian, and Mark Crowley.  ``Training Cooperative Agents for Multi-Agent Reinforcement Learning'', \emph{}. In \emph{Proc. of the 18th International Conference on Autonomous Agents and Multiagent Systems (AAMAS 2019)}. (),  pages. Montreal, Canada, 2019.
    }%
}

\textbf{Abstract:} Deep Learning and back-propagation has been successfully used to perform centralized training with communication protocols among multiple agents in a cooperative environment. In this paper we present techniques for centralized training of Multi-Agent (Deep) Reinforcement Learning (MARL) using the model-free Deep Q-Network as the baseline model and message sharing between agents. We present a novel, scalable, centralized MARL training technique, which separates the message learning module from the policy module. The separation of these modules helps in faster convergence in complex domains like autonomous driving simulators. A second contribution uses the centrally trained model to bootstrap training of distributed, independent, cooperative agent policies for execution and thus addresses the challenges of noise and communication bottlenecks in real-time communication channels. This paper theoretically and empirically compares our centralized training algorithms to current research in the field of MARL. We also present and release a new OpenAI-Gym environment which can be used for multi-agent research as it simulates multiple autonomous cars driving cooperatively on a highway.

\textbf{paperdesc:} This paper introduces a novel method for learning communication protocols between partially co-operative agents in simulation. It arose out of complementary research between two of my students which blossomed into it's own new result. This conference is the premier international venue for multi-agent research using AI. I worked closely with the students at every step of the development, experimentation and wrote large parts of the paper.

\textbf{Description:} 



\newpage
\subsubsection{\textbf{SciKnow-2019} : Semantic Workflows and Machine Learning for the Assessment of Carbon Storage by Urban Trees}
%garijo2019sciknow
%status: 1
\begin{itemize}
\item \textbf{keywords:} Machine Learning, Computational Sustainability, semantics, showcase
\item \textbf{Arxiv:}  (if any)
\end{itemize}
%todo: type (conference, journal, ptatne...)
%todo: can we check if a field is empy?
%todo: copy all paperdesc text from cv to bibtex in this way
%todo: create another version of this template for markdown, with a page separator

\fbox{%
\parbox{\textwidth}{%
        \textbf{CITATION: \texttt{garijo2019sciknow} \cite{garijo2019sciknow}}:\\
            Juan Carrillo, Daniel Garijo, Mark Crowley, Yolanda Gil, and Katherine Borda.  ``Semantic Workflows and Machine Learning for the Assessment of Carbon Storage by Urban Trees'', \emph{}. In \emph{Third International Workshop on Capturing Scientific Knowledge (Sciknow 2019), Collocated with the tenth International Conference on Knowledge Capture (K-CAP)}. (), 6 pages. Los Angeles, California, USA, 2019.
    }%
}

\textbf{Abstract:} 

\textbf{paperdesc:} This paper was a report on work carried out over the summer by Carrillo during a Mitacs intership with Prof. Yolanda Gill's lab, it was written by the student based on collaboration and input from all the other authors using their data and workflows and with myself advising on Machine Learning models and validation methods which were appropriate to use.

\textbf{Description:} 



\newpage
\subsubsection{\textbf{TAC-ITS-2019} : Comparison of Deep Learning models for Determining Road Surface Condition from Roadside Camera Images and Weather Data}
%carrillo2019tac
%status: 1
\begin{itemize}
\item \textbf{keywords:} Machine Learning,smart cities
\item \textbf{Arxiv:}  (if any)
\end{itemize}
%todo: type (conference, journal, ptatne...)
%todo: can we check if a field is empy?
%todo: copy all paperdesc text from cv to bibtex in this way
%todo: create another version of this template for markdown, with a page separator

\fbox{%
\parbox{\textwidth}{%
        \textbf{CITATION: \texttt{carrillo2019tac} \cite{carrillo2019tac}}:\\
            J. Carrillo, M. Crowley, G. Pan, and L. Fu.  ``Comparison of Deep Learning models for Determining Road Surface Condition from Roadside Camera Images and Weather Data'', \emph{}. In \emph{TAC-ITS Canada Joint Conference}. (), 17 pages. Halifax, Canada, 2019.
    }%
}

\textbf{Abstract:} Road maintenance during the Winter season is a safety critical and resource demanding operation. One of its key activities is determining road surface condition (RSC) in order to prioritize roads and allocate cleaning efforts such as plowing or salting. Two conventional approaches for determining RSC are: visual examination of roadside camera images by trained personnel and patrolling the roads to perform on-site inspections. However, with more than 500 cameras collecting images across Ontario, visual examination becomes a resource-intensive activity, difficult to scale especially during periods of snow storms. This paper presents the preliminary results of an ongoing study focused on improving the efficiency of road maintenance operations. We use multiple Deep Learning models to automatically determine RSC from roadside camera images and weather variables, extending previous research where similar methods have been used to deal with the problem. The dataset we use was collected during the 2017-2018 Winter season from 40 stations connected to the Ontario Road Weather Information System (RWIS), it includes 14.000 labelled images and 70.000 weather measurements. In particular, we train and evaluate the performance of seven state-of-the-art models from the Computer Vision literature, including the recent DenseNet, NASNet, and MobileNet. Also, by integrating observations from weather variables, the models are able to better ascertain RSC under poor visibility conditions.

\textbf{paperdesc:} This paper was primarily authored by my MASc student Juan Carrillo under my guidance. The data and domain came from a previous study from Prof. Liping Fu's group and his student Guangyuan Pan helped us use their existing data and make fair comparisons to their previous study.

\textbf{Description:} 



\newpage
\subsubsection{\textbf{CanAI-2019} : Instance Ranking and Numerosity Reduction Using Matrix Decomposition and Subspace Learning}
%ghojogh2019ccai
%status: 1
\begin{itemize}
\item \textbf{keywords:} Manifold-Learning, Data Reduction, Numerosity Reduction, showcase
\item \textbf{Arxiv:}  (if any)
\end{itemize}
%todo: type (conference, journal, ptatne...)
%todo: can we check if a field is empy?
%todo: copy all paperdesc text from cv to bibtex in this way
%todo: create another version of this template for markdown, with a page separator

\fbox{%
\parbox{\textwidth}{%
        \textbf{CITATION: \texttt{ghojogh2019ccai} \cite{ghojogh2019ccai}}:\\
            Benyamin Ghojogh and Mark Crowley.  ``Instance Ranking and Numerosity Reduction Using Matrix Decomposition and Subspace Learning'', \emph{}. In \emph{Canadian Conference on Artificial Intelligence}. (), 12 pages. Kingston, ON, Canada, 2019.
    }%
}

\textbf{Abstract:} One way to deal with the ever increasing amount of available data for processing is to rank data instances by usefulness and reduce the dataset size. In this work, we introduce a framework to achieve this using matrix decomposition and subspace learning. Our central contribution is a novel similarity measure for data instances that uses the basis obtained from matrix decomposition of the dataset. Using this similarity measure, we propose several related algorithms for ranking data instances and performing numerosity reduction. We then validate the effectiveness of these algorithms for data reduction on several datasets for classification, regression, and clustering tasks.

\textbf{paperdesc:} 

\textbf{Description:} 



\newpage
\subsubsection{\textbf{CARSP-2019} : Integration of Roadside Camera Images and Weather Data for monitoring Winter Road Surface Conditions}
%carrillo2019carsp
%status: 1
\begin{itemize}
\item \textbf{keywords:} Deep Neural Networks,Automotive,CNNs,Classification,Urban Infrastucture,Machine Learning,smart cities
\item \textbf{Arxiv:}  (if any)
\end{itemize}
%todo: type (conference, journal, ptatne...)
%todo: can we check if a field is empy?
%todo: copy all paperdesc text from cv to bibtex in this way
%todo: create another version of this template for markdown, with a page separator

\fbox{%
\parbox{\textwidth}{%
        \textbf{CITATION: \texttt{carrillo2019carsp} \cite{carrillo2019carsp}}:\\
            Juan Carrillo and Mark Crowley.  ``Integration of Roadside Camera Images and Weather Data for monitoring Winter Road Surface Conditions'', \emph{}. In \emph{Canadian Association of Road Safety Professionals (CARSP) Conference}. (), 4 pages. Calgary, Canada, 2019.
    }%
}

\textbf{Abstract:} Background/Context: During the Winter season, real-time monitoring of road surface conditions is critical for the safety of drivers and road maintenance operations. Previous research has evaluated the potential of image classification methods for detecting road snow coverage by processing images from roadside cameras installed in RWIS (Road Weather Information System) stations. However, it is a challenging task due to limitations such as image resolution, camera angle, and illumination. Two common approaches to improve the accuracy of image classification methods are: adding more input features to the model and increasing the number of samples in the training dataset. Additional input features can be weather variables and more sample images can be added by including other roadside cameras. Although RWIS stations are equipped with both cameras and weather measurement instruments, they are only a subset of the total number of roadside cameras installed across transportation networks, most of which do not have weather measurement instruments. Thus, improvements in use of image data could benefit from additional data sources. Aims/Objectives: The first objective of this study is to complete an exploratory data analysis over three data sources in Ontario: RWIS stations, all the other MTO (Ministry of Transportation of Ontario) roadside cameras, and Environment Canada weather stations. The second objective is to determine the feasibility of integrating these three datasets into a more extensive and richer dataset with weather variables as additional features and other MTO roadside cameras as additional sources of images. Methods/Targets: First, we quantify the advantage of adding other MTO roadside cameras using spatial statistics, the number of monitored roads, and the coverage of ecoregions with different climate regimes. We then analyze experimental variograms from the literature and determine the feasibility of using Environment Canada stations and RWIS stations to interpolate weather variables for all the other MTO roadside cameras without weather instruments. Results/Activities: By adding all other MTO cameras as image data sources, the total number of cameras in the dataset increases from 139 to 578 across Ontario. The average distance to the nearest camera decreases from 38.4km to 9.4km, and the number of monitored roads increases approximately four times. Additionally, six times more cameras are available in the four most populated ecoregions in Ontario. The experimental variograms show that it is feasible to interpolate weather variables with reasonable accuracy. Moreover, observations in the three datasets are collected with similar frequency, which facilitates our data integration approach. Discussion/Deliverables: Integrating these three datasets is feasible and can benefit the design and development of automated image classification methods for monitoring road snow coverage. We do not consider data from pavement-embedded sensors, an additional line of research may explore the integration of this data. Our approach can provide actionable insights which can be used to more selectively perform manual patrolling to better identify road surface conditions. Conclusions: Our initial results are promising and demonstrate that additional, image only datasets can be added to road monitoring data by using existing multimodal sensors as ground truth, which will lead to greater performance on the future image classification tasks.

\textbf{paperdesc:} 

\textbf{Description:} 



\subsection{Showcased Papers}
\newpage
\subsubsection{\textbf{CanAI-2021} : Active Measure Reinforcement Learning for Observation Cost Minimization: A framework for minimizing measurement costs in reinforcement learning}
%bellinger2021canai
%status: 1
\begin{itemize}
\item \textbf{keywords:} reinforcement-learning, Machine-Learning, chemistry, proj-deepchemrl, showcase
\item \textbf{Arxiv:} 2005.12697 (if any)
\end{itemize}
%todo: type (conference, journal, ptatne...)
%todo: can we check if a field is empy?
%todo: copy all paperdesc text from cv to bibtex in this way
%todo: create another version of this template for markdown, with a page separator

\fbox{%
\parbox{\textwidth}{%
        \textbf{CITATION: \texttt{bellinger2021canai} \cite{bellinger2021canai}}:\\
            Colin Bellinger, Rory Coles, Mark Crowley, and Isaac Tamblyn.  ``Active Measure Reinforcement Learning for Observation Cost Minimization: A framework for minimizing measurement costs in reinforcement learning'', \emph{}. In \emph{Canadian Conference on Artificial Intelligence}. (), 12 pages. , 2021.
    }%
}

\textbf{Abstract:} Markov Decision Processes (MDP) with explicit measurement cost are a class of en- vironments in which the agent learns to maximize the costed return. Here, we define the costed return as the discounted sum of rewards minus the sum of the explicit cost of measuring the next state. The RL agent can freely explore the relationship between actions and rewards but is charged each time it measures the next state. Thus, an op- timal agent must learn a policy without making a large number of measurements. We propose the active measure RL framework (Amrl) as a solution to this novel class of problem, and contrast it with standard reinforcement learning under full observability and planning under partially observability. We demonstrate that Amrl-Q agents learn to shift from a reliance on costly measurements to exploiting a learned transition model in order to reduce the number of real-world measurements and achieve a higher costed return. Our results demonstrate the superiority of Amrl-Q over standard RL methods, Q-learning and Dyna-Q, and POMCP for planning under a POMDP in environments with explicit measurement costs.

\textbf{paperdesc:} Bellinger and Tamblyn are researchers at NRC. They carried out some experiments based on my idea of extending a standard RL approach to include observation costs, whcih is necessary to enable automated control of scientific experiments. This work arises from the NRC-UW collaboration grant I lead. This builds on work from \cite{bellinger2020reinforcement} last year.

\textbf{Description:} 



\newpage
\subsubsection{\textbf{AAMAS-2021} : Partially Observable Mean Field Reinforcement Learning}
%ganapathisubramanian2021aamas
%status: 1
\begin{itemize}
\item \textbf{keywords:} marl, reinforcement-learning, mean field theory, partial observation, showcase
\item \textbf{Arxiv:} 2012.15791 (if any)
\end{itemize}
%todo: type (conference, journal, ptatne...)
%todo: can we check if a field is empy?
%todo: copy all paperdesc text from cv to bibtex in this way
%todo: create another version of this template for markdown, with a page separator

\fbox{%
\parbox{\textwidth}{%
        \textbf{CITATION: \texttt{ganapathisubramanian2021aamas} \cite{ganapathisubramanian2021aamas}}:\\
            Sriram Ganapathi Subramanian, Matthew Taylor, Mark Crowley, and Pascal Poupart.  ``Partially Observable Mean Field Reinforcement Learning'', \emph{}. In \emph{Proceedings of the 20th International Conference on Autonomous Agents and MultiAgent Systems (AAMAS)}. (), 537-545 pages. London, United Kingdom, 2021.
    }%
}

\textbf{Abstract:} Traditional multi-agent reinforcement learning algorithms are not scalable to environments with more than a few agents, since these algorithms are exponential in the number of agents. Recent research has introduced successful methods to scale multi-agent reinforcement learning algorithms to many agent scenarios using mean field theory. Previous work in this field assumes that an agent has access to exact cumulative metrics regarding the mean field behaviour of the system, which it can then use to take its actions. In this paper, we relax this assumption and maintain a distribution to model the uncertainty regarding the mean field of the system. We consider two different settings for this problem. In the first setting, only agents in a fixed neighbourhood are visible, while in the second setting, the visibility of agents is determined at random based on distances. For each of these settings, we introduce a Q-learning based algorithm that can learn effectively. We prove that this Q-learning estimate stays very close to the Nash Q-value (under a common set of assumptions) for the first setting. We also empirically show our algorithms outperform multiple baselines in three different games in the MAgents framework, which supports large environments with many agents learning simultaneously to achieve possibly distinct goals.

\textbf{paperdesc:} This paper submission extends work which my PhD student Ganapathi Subramanian has done previously while hired as an industry co-op term with Poupart. The core idea came from the student himself. The writing and theory was a full collaboration between all contributors through weekly calls and collaborative writing.

\textbf{Description:} 



\newpage
\subsubsection{\textbf{ICPR-2021} : Batch-Incremental Triplet Sampling for Training Triplet Networks Using Bayesian Updating Theorem}
%sikaroudi2021icpr
%status: 1
\begin{itemize}
\item \textbf{keywords:} showcase
\item \textbf{Arxiv:} 2007.05610 (if any)
\end{itemize}
%todo: type (conference, journal, ptatne...)
%todo: can we check if a field is empy?
%todo: copy all paperdesc text from cv to bibtex in this way
%todo: create another version of this template for markdown, with a page separator

\fbox{%
\parbox{\textwidth}{%
        \textbf{CITATION: \texttt{sikaroudi2021icpr} \cite{sikaroudi2021icpr}}:\\
            Milad Sikaroudi, Benyamin Ghojogh, Fakhri Karray, Mark Crowley, and H. R. Tizhoosh.  ``Batch-Incremental Triplet Sampling for Training Triplet Networks Using Bayesian Updating Theorem'', \emph{}. In \emph{25th International Conference on Pattern Recognition (ICPR)}. (), 7 pages. Milan, Italy (virtual), 2021.
    }%
}

\textbf{Abstract:} Variants of Triplet networks are robust entities for learning a discriminative embedding subspace. There exist different triplet mining approaches for selecting the most suitable training triplets. Some of these mining methods rely on the extreme distances between instances, and some others make use of sampling. However, sampling from stochastic distributions of data rather than sampling merely from the existing embedding instances can provide more discriminative information. In this work, we sample triplets from distributions of data rather than from existing instances. We consider a multivariate normal distribution for the embedding of each class. Using Bayesian updating and conjugate priors, we update the distributions of classes dynamically by receiving the new mini-batches of training data. The proposed triplet mining with Bayesian updating can be used with any triplet-based loss function, e.g., triplet-loss or Neighborhood Component Analysis (NCA) loss. Accordingly, Our triplet mining approaches are called Bayesian Updating Triplet (BUT) and Bayesian Updating NCA (BUNCA), depending on which loss function is being used. Experimental results on two public datasets, namely MNIST and histopathology colorectal cancer (CRC), substantiate the effectiveness of the proposed triplet mining method.

\textbf{paperdesc:} 

\textbf{Description:} 



\newpage
\subsubsection{\textbf{CanAI-2020} : Reinforcement Learning in a Physics-Inspired Semi-Markov Environment}
%bellinger2020reinforcement
%status: 1
\begin{itemize}
\item \textbf{keywords:} Reinforcement-Learning, proj-deepchemrl, deep learning, showcase
\item \textbf{Arxiv:}  (if any)
\end{itemize}
%todo: type (conference, journal, ptatne...)
%todo: can we check if a field is empy?
%todo: copy all paperdesc text from cv to bibtex in this way
%todo: create another version of this template for markdown, with a page separator

\fbox{%
\parbox{\textwidth}{%
        \textbf{CITATION: \texttt{bellinger2020reinforcement} \cite{bellinger2020reinforcement}}:\\
            Colin Bellinger, Rory Coles, Mark Crowley, and Isaac Tamblyn.  ``Reinforcement Learning in a Physics-Inspired Semi-Markov Environment'', \emph{}. In \emph{Canadian Conference on Artificial Intelligence}. 12109(), 55-66 pages. Ottawa, Canada (virtual), 2020.
    }%
}

\textbf{Abstract:} 

\textbf{paperdesc:} I was intimately involved in creation of the core idea of this paper and was involved in writing, development and multiple revisions. It is resulting from a post doctoral fellow (Bellinger) working under Tamblyn at NRC on our joint project using Reinforcement Learning for Material Design and Discovery.

\textbf{Description:} 



\newpage
\subsubsection{\textbf{EMBC-2020} : Supervision and Source Domain Impact on Representation Learning: A Histopathology Case Study}
%sikaroudi2020embc
%status: 1
\begin{itemize}
\item \textbf{keywords:} representation-learning, proj-digipath, proj-digipath, proj-digipath, medical, showcase
\item \textbf{Arxiv:}  (if any)
\end{itemize}
%todo: type (conference, journal, ptatne...)
%todo: can we check if a field is empy?
%todo: copy all paperdesc text from cv to bibtex in this way
%todo: create another version of this template for markdown, with a page separator

\fbox{%
\parbox{\textwidth}{%
        \textbf{CITATION: \texttt{sikaroudi2020embc} \cite{sikaroudi2020embc}}:\\
            Milad Sikaroudi, Amir Safarpoor, Benyamin Ghojogh, Sobhan Shafiei, Mark Crowley, and HR Tizhoosh.  ``Supervision and Source Domain Impact on Representation Learning: A Histopathology Case Study'', \emph{}. In \emph{International Conference of the IEEE Engineering in Medicine and Biology Society (EMBC'20)}. (), 4 pages. Montreal, Quebec, Canada (virtual), 2020.
    }%
}

\textbf{Abstract:} As many algorithms depend on a suitable representation of data, learning
unique features is considered a crucial task. Although supervised techniques
using deep neural networks have boosted the performance of representation
learning, the need for a large set of labeled data limits the application of
such methods. As an example, high-quality delineations of regions of interest
in the field of pathology is a tedious and time-consuming task due to the large
image dimensions. In this work, we explored the performance of a deep neural
network and triplet loss in the area of representation learning. We
investigated the notion of similarity and dissimilarity in pathology
whole-slide images and compared different setups from unsupervised and
semi-supervised to supervised learning in our experiments. Additionally,
different approaches were tested, applying few-shot learning on two publicly
available pathology image datasets. We achieved high accuracy and
generalization when the learned representations were applied to two different
pathology datasets

\textbf{paperdesc:} IJCNN is very well respected conference on neural networks held as part of the IEEE World Congress on Computational Intelligence (WCCI). This paper expands on the new trend of methods utilizing variable triplets to build "Siamese" deep nerural networks, in this paper we provide an improved method for training such models which is particularly useful for complex image understanding.

\textbf{Description:} 



\newpage
\subsubsection{\textbf{CDC-2020} : Distributed Nonlinear Model Predictive Control and Metric Learning for Heterogeneous Vehicle Platooning with Cut-in/Cut-out Maneuvers}
%basiri2020cdc
%status: 1
\begin{itemize}
\item \textbf{keywords:} showcase
\item \textbf{Arxiv:}  (if any)
\end{itemize}
%todo: type (conference, journal, ptatne...)
%todo: can we check if a field is empy?
%todo: copy all paperdesc text from cv to bibtex in this way
%todo: create another version of this template for markdown, with a page separator

\fbox{%
\parbox{\textwidth}{%
        \textbf{CITATION: \texttt{basiri2020cdc} \cite{basiri2020cdc}}:\\
            Mohammad Hossein Basiri, Benyamin Ghojogh, Nasser L Azad, Sebastian Fischmeister, Fakhri Karray, and Mark Crowley.  ``Distributed Nonlinear Model Predictive Control and Metric Learning for Heterogeneous Vehicle Platooning with Cut-in/Cut-out Maneuvers'', \emph{arXiv preprint arXiv:2004.00417}. In \emph{Proceeding of the 59th IEEE Conference on Decision and Control (CDC-2020)}. (), 2849-2856 pages. Jeju Island, Korea (virtual), 2020.
    }%
}

\textbf{Abstract:} 

\textbf{paperdesc:} This collaboration paper deals with platoon driving, combining use of control algorithms with metric learning. I worked with my student Ghojogh on the theoretical metric learning component and was involved heavily in paper editing and revision.

\textbf{Description:} 



\newpage
\subsubsection{\textbf{SMC-2020} : Isolation Mondrian Forest for Batch and Online Anomaly Detection}
%ma2020imondrian
%status: 1
\begin{itemize}
\item \textbf{keywords:} showcase
\item \textbf{Arxiv:}  (if any)
\end{itemize}
%todo: type (conference, journal, ptatne...)
%todo: can we check if a field is empy?
%todo: copy all paperdesc text from cv to bibtex in this way
%todo: create another version of this template for markdown, with a page separator

\fbox{%
\parbox{\textwidth}{%
        \textbf{CITATION: \texttt{ma2020imondrian} \cite{ma2020imondrian}}:\\
            Haoran Ma, Benyamin Ghojogh, Maria N Samad, Dongyu Zheng, and Mark Crowley.  ``Isolation Mondrian Forest for Batch and Online Anomaly Detection'', \emph{}. In \emph{IEEE International Conference on Systems, Man, and Cybernetics (IEEE-SMC-2020)}. (), 7 pages. Toronto, Canada (virtual), 2020.
    }%
}

\textbf{Abstract:} 

\textbf{paperdesc:} This is a novel anomaly detection algorithm which I came up with and put forward to graduate student Samad and was then fleshed out and expanded with the other students. I was centrally involved in all aspects of this paper, including writing and I also presented it at the virtual conference. The algorithm fuses two ideas, "isolation" from ensemble trees methods for anomaly detection and "Mondrian forests" which can learn flexible regression models from streaming data.

\textbf{Description:} This is a novel anomaly detection algorithm which I came up with and put forward to graduate student Samad and was then fleshed out and expanded with the other students. I was centrally involved in all aspects of this paper, including writing and I also presented it at the virtual conference. 
    The algorithm fuses two ideas, "isolation" from ensemble trees methods for anomaly detection and "Mondrian forests" which can learn flexible regression models from streaming data.



\newpage
\subsubsection{\textbf{EnvRevJrnl-2020} : A review of machine learning applications in wildfire science and management}
%jain2020review
%status: 1
\begin{itemize}
\item \textbf{keywords:} showcase
\item \textbf{Arxiv:} 2003.00646 (if any)
\end{itemize}
%todo: type (conference, journal, ptatne...)
%todo: can we check if a field is empy?
%todo: copy all paperdesc text from cv to bibtex in this way
%todo: create another version of this template for markdown, with a page separator

\fbox{%
\parbox{\textwidth}{%
        \textbf{CITATION: \texttt{jain2020review} \cite{jain2020review}}:\\
            Piyush Jain, Sean CP Coogan, Sriram Ganapathi Subramanian, Mark Crowley, Steve Taylor, and Mike D Flannigan.  ``A review of machine learning applications in wildfire science and management'', \emph{Environmental Reviews}. In \emph{}. 28(3),  pages. , 2020.
    }%
}

\textbf{Abstract:} 

\textbf{paperdesc:} This review paper arose from my visit to BC to speak on the use of AI for Forest Fire management and led to a collaboration amongst these senior researchers, myself and my PhD student Sriram. We collaborated on every part of the paper, I especially wrote the general AI/ML background and checked that each specific Forest Fire domain was connected correctly to the ML literature. It will serve as a much-needed resource for researches in my field as well as applied Forest Fire Management and Science fields.

\textbf{Description:} 



\newpage
\subsubsection{\textbf{ICMI-2020} : Recognition of a Robot's Affective Expressions under Conditions with Limited Visibility}
%ghafurian2020icmi
%status: 1
\begin{itemize}
\item \textbf{keywords:} showcase, human-robot interaction
\item \textbf{Arxiv:}  (if any)
\end{itemize}
%todo: type (conference, journal, ptatne...)
%todo: can we check if a field is empy?
%todo: copy all paperdesc text from cv to bibtex in this way
%todo: create another version of this template for markdown, with a page separator

\fbox{%
\parbox{\textwidth}{%
        \textbf{CITATION: \texttt{ghafurian2020icmi} \cite{ghafurian2020icmi}}:\\
            Moojan Ghafurian, Sami Alperen Akgun, Mark Crowley, and Kerstin Dautenhahn.  ``Recognition of a Robot's Affective Expressions under Conditions with Limited Visibility'', \emph{}. In \emph{International Conference on Human-Robot Interaction}. (),  pages. Cambridge, UK (virtual), 2020.
    }%
}

\textbf{Abstract:} The capability of showing affective expressions is important for the design of social robots in many contexts, especially where the robot is designed to communicate with humans. It is reasonable to expect that, similar to all other interaction modalities, communicating with affective expressions is not without limitations. In this paper, we present two online video studies (N=72 and N=50) and investigate if/how much the recognition of affective displays of a zoomorphic robot is affected under situations with different levels of visibility. Five affective expressions combined with five visibility effects were studied. The intensity of the effects was more pronounced in the second experiment. While visual constraints affected recognition of expressions, our results showed that affective displays of the robot conveyed through its head and body gestures can be robust and recognition rates can still be high even under severe constraints. This study supported the effectiveness of using affective displays as a complementary communication modality in human-robot interaction in situations with low visibility.

\textbf{paperdesc:} 

\textbf{Description:} 



\newpage
\subsubsection{\textbf{CanAI-2020} : Deep Multi Agent Reinforcement Learning for Autonomous Driving}
%bhalla2020deep
%status: 1
\begin{itemize}
\item \textbf{keywords:} showcase, MARL
\item \textbf{Arxiv:}  (if any)
\end{itemize}
%todo: type (conference, journal, ptatne...)
%todo: can we check if a field is empy?
%todo: copy all paperdesc text from cv to bibtex in this way
%todo: create another version of this template for markdown, with a page separator

\fbox{%
\parbox{\textwidth}{%
        \textbf{CITATION: \texttt{bhalla2020deep} \cite{bhalla2020deep}}:\\
            Sushrut Bhalla, Sriram Ganapathi Subramanian, and Mark Crowley.  ``Deep Multi Agent Reinforcement Learning for Autonomous Driving'', \emph{}. In \emph{Canadian Conference on Artificial Intelligence}. (), 67--78 pages. , 2020.
    }%
}

\textbf{Abstract:} 

\textbf{paperdesc:} 

\textbf{Description:} 



\newpage
\subsubsection{\textbf{SciKnow-2019} : Semantic Workflows and Machine Learning for the Assessment of Carbon Storage by Urban Trees}
%garijo2019sciknow
%status: 1
\begin{itemize}
\item \textbf{keywords:} Machine Learning, Computational-Sustainability, semantics, showcase
\item \textbf{Arxiv:}  (if any)
\end{itemize}
%todo: type (conference, journal, ptatne...)
%todo: can we check if a field is empy?
%todo: copy all paperdesc text from cv to bibtex in this way
%todo: create another version of this template for markdown, with a page separator

\fbox{%
\parbox{\textwidth}{%
        \textbf{CITATION: \texttt{garijo2019sciknow} \cite{garijo2019sciknow}}:\\
            Juan Carrillo, Daniel Garijo, Mark Crowley, Yolanda Gil, and Katherine Borda.  ``Semantic Workflows and Machine Learning for the Assessment of Carbon Storage by Urban Trees'', \emph{}. In \emph{Third International Workshop on Capturing Scientific Knowledge (Sciknow 2019), Collocated with the tenth International Conference on Knowledge Capture (K-CAP)}. (), 6 pages. Los Angeles, California, USA, 2019.
    }%
}

\textbf{Abstract:} 

\textbf{paperdesc:} This paper was a report on work carried out over the summer by Carrillo during a Mitacs intership with Prof. Yolanda Gill's lab, it was written by the student based on collaboration and input from all the other authors using their data and workflows and with myself advising on Machine Learning models and validation methods which were appropriate to use.

\textbf{Description:} 



\newpage
\subsubsection{\textbf{CanAI-2019} : Instance Ranking and Numerosity Reduction Using Matrix Decomposition and Subspace Learning}
%ghojogh2019ccai
%status: 1
\begin{itemize}
\item \textbf{keywords:} Manifold-Learning, Data Reduction, Numerosity Reduction, showcase
\item \textbf{Arxiv:}  (if any)
\end{itemize}
%todo: type (conference, journal, ptatne...)
%todo: can we check if a field is empy?
%todo: copy all paperdesc text from cv to bibtex in this way
%todo: create another version of this template for markdown, with a page separator

\fbox{%
\parbox{\textwidth}{%
        \textbf{CITATION: \texttt{ghojogh2019ccai} \cite{ghojogh2019ccai}}:\\
            Benyamin Ghojogh and Mark Crowley.  ``Instance Ranking and Numerosity Reduction Using Matrix Decomposition and Subspace Learning'', \emph{}. In \emph{Canadian Conference on Artificial Intelligence}. (), 12 pages. Kingston, ON, Canada, 2019.
    }%
}

\textbf{Abstract:} One way to deal with the ever increasing amount of available data for processing is to rank data instances by usefulness and reduce the dataset size. In this work, we introduce a framework to achieve this using matrix decomposition and subspace learning. Our central contribution is a novel similarity measure for data instances that uses the basis obtained from matrix decomposition of the dataset. Using this similarity measure, we propose several related algorithms for ranking data instances and performing numerosity reduction. We then validate the effectiveness of these algorithms for data reduction on several datasets for classification, regression, and clustering tasks.

\textbf{paperdesc:} 

\textbf{Description:} 



\newpage
\subsubsection{\textbf{ICIAR-2019} : Principal Component Analysis Using Structural Similarity Index for Images}
%ghojogh2019pcassim
%status: 1
\begin{itemize}
\item \textbf{keywords:} Manifold-Learning, Data Reduction, Numerosity Reduction, showcase
\item \textbf{Arxiv:}  (if any)
\end{itemize}
%todo: type (conference, journal, ptatne...)
%todo: can we check if a field is empy?
%todo: copy all paperdesc text from cv to bibtex in this way
%todo: create another version of this template for markdown, with a page separator

\fbox{%
\parbox{\textwidth}{%
        \textbf{CITATION: \texttt{ghojogh2019pcassim} \cite{ghojogh2019pcassim}}:\\
            Benyamin Ghojogh, Fakhri Karray, and Mark Crowley.  ``Principal Component Analysis Using Structural Similarity Index for Images'', \emph{}. In \emph{International Conference on Image Analysis and Recognition (ICIAR-19)}. (), 77--88 pages. Waterloo, Canada, 2019.
    }%
}

\textbf{Abstract:} 

\textbf{paperdesc:} 

\textbf{Description:} 



\newpage
\subsubsection{\textbf{CanAI-2018} : Combining MCTS and A3C for prediction of spatially spreading processes in forest wildfire settings}
%Subramanian2018CCAI
%status: 1
\begin{itemize}
\item \textbf{keywords:} A3C,Computational-Sustainability,Wildfire-Management,Monte-Carlo Tree Search,Reinforcement-Learning, showcase
\item \textbf{Arxiv:}  (if any)
\end{itemize}
%todo: type (conference, journal, ptatne...)
%todo: can we check if a field is empy?
%todo: copy all paperdesc text from cv to bibtex in this way
%todo: create another version of this template for markdown, with a page separator

\fbox{%
\parbox{\textwidth}{%
        \textbf{CITATION: \texttt{Subramanian2018CCAI} \cite{Subramanian2018CCAI}}:\\
            Sriram Ganapathi Subramanian and Mark Crowley.  ``Combining MCTS and A3C for prediction of spatially spreading processes in forest wildfire settings'', \emph{}. In \emph{Canadian Conference on Artificial Intelligence}. 10832 LNAI(), 285--291 pages. Toronto, Ontario, Canada, 2018.
    }%
}

\textbf{Abstract:} In recent years, Deep Reinforcement Learning (RL) algorithms have shown super-human performance in a variety Atari and classic board games like chess and GO. Research into applications of RL in other domains with spatial considerations like environmental planning are still in their nascent stages. In this paper, we introduce a novel combination of Monte-Carlo Tree Search (MCTS) and A3C algorithms on an online simulator of a wildfire, on a pair of forest fires in Northern Alberta (Fort McMurray and Richardson fires) and on historical Saskatchewan fires previously compared by others to a physics-based simulator. We conduct several experiments to predict fire spread for several days before and after the given spatial information of fire spread and ignition points. Our results show that the advancements in Deep RL applications in the gaming world have advantages in spatially spreading real-world problems like forest fires. {\textcopyright} Springer International Publishing AG, part of Springer Nature 2018.

\textbf{paperdesc:} I worked closely with my MASc student Subramanian on every aspect of this paper which grew out of previous publications we had with empirical results on this topic, resulting in a new combined model fusing the best of multiple approaches.

\textbf{Description:} 



\newpage
\subsubsection{\textbf{-2018} : Using Spatial Reinforcement Learning to Build Forest Wildfire Dynamics Models From Satellite Images}
%Subramanian2017a
%status: 1
\begin{itemize}
\item \textbf{keywords:} A3C,deep learning, Wildfire-Management,Machine Learning,Reinforcement-Learning,spatially spreading processes,sustainabtility, showcase
\item \textbf{Arxiv:}  (if any)
\end{itemize}
%todo: type (conference, journal, ptatne...)
%todo: can we check if a field is empy?
%todo: copy all paperdesc text from cv to bibtex in this way
%todo: create another version of this template for markdown, with a page separator

\fbox{%
\parbox{\textwidth}{%
        \textbf{CITATION: \texttt{Subramanian2017a} \cite{Subramanian2017a}}:\\
            Sriram Ganapathi Subramanian and Mark Crowley.  ``Using Spatial Reinforcement Learning to Build Forest Wildfire Dynamics Models From Satellite Images'', \emph{Frontiers in ICT}. In \emph{}. 5(), 6 pages. , 2018.
    }%
}

\textbf{Abstract:} Machine learning algorithms have increased tremendously in power in recent years but have yet to be fully utilized in many ecology and sustainable resource management domains such as wildlife reserve design, forest fire management and invasive species spread. One thing these domains have in common is that they contain dynamics that can be characterized as a Spatially Spreading Process (SSP) which requires many parameters to be set precisely to model the dynamics, spread rates and directional biases of the elements which are spreading. We present related work in Artificial Intelligence and Machine Learning for SSP sustainability domains including forest wildfire prediction. We then introduce a novel approach for learning in SSP domains using Reinforcement Learning (RL) where fire is the agent at any cell in the landscape and the set of actions the fire can take from a location at any point in time includes spreading North, South, East, West or not spreading. This approach inverts the usual RL setup since the dynamics of the corresponding Markov Decision Process (MDP) is a known function for immediate wildfire spread. Meanwhile, we learn an agent policy for a predictive model of the dynamics of a complex spatially-spreading process. Rewards are provided for correctly classifying which cells are on fire or not compared to satellite and other related data. We examine the behaviour of five RL algorithms on this problem: Value Iteration, Policy Iteration, Q-Learning, Monte Carlo Tree Search and Asynchronous Advantage Actor-Critic (A3C). We compare to a Gaussian process based supervised learning approach and discuss the relation of our approach to manually constructed, state-of-the-art methods from forest wildfire modelling. We also discuss the relation of our approach to manually constructed, state-of-the-art methods from forest wildfire modelling. We validate our approach with satellite image data of two massive wildfire events in Northern Alberta, Canada, the Fort McMurray fire of 2016 and the Richardson fire of 2011. The results show that we can learn predictive, agent-based policies as models of spatial dynamics using RL on readily available satellite images that other methods and have many additional advantages in terms of generalizability and interpretability.

\textbf{paperdesc:} I wrote the the majority of the test in collaboration with my MASc student at the time who provided experimental results and figures. Frontiers is a peer reviewed, online, open access journal. This was a special topic, spearheaded by a Canadian government researchers at Agriculture and Agri-Food Canada, focussing on collaborative projects by researchers in government and academia related to use of computing to advance knowledge and abilities in the environmental policy domain.

\textbf{Description:} 



\newpage
\subsubsection{\textbf{-2017} : AI Education Through Real World Problems}
%Crowley2017
%status: 1
\begin{itemize}
\item \textbf{keywords:} showcase
\item \textbf{Arxiv:}  (if any)
\end{itemize}
%todo: type (conference, journal, ptatne...)
%todo: can we check if a field is empy?
%todo: copy all paperdesc text from cv to bibtex in this way
%todo: create another version of this template for markdown, with a page separator

\fbox{%
\parbox{\textwidth}{%
        \textbf{CITATION: \texttt{Crowley2017} \cite{Crowley2017}}:\\
            Mark Crowley.  ``AI Education Through Real World Problems'', \emph{}. In \emph{The Seventh Symposium on Educational Advances in Artificial Intellgience.}. (),  pages. San Francisco, USA., 2017.
    }%
}

\textbf{Abstract:} 

\textbf{paperdesc:} Automation of safety-critical systems is one of the most important and challenging areas of AI/ML research, I presented my vision for integrating engineering principles into AI education in a unified way.

\textbf{Description:} 



\newpage
\subsubsection{\textbf{RLDM-2017} : Learning Forest Wildfire Dynamics from Satellite Images Using Reinforcement Learning}
%Subramanian2017
%status: 1
\begin{itemize}
\item \textbf{keywords:} A2C, reinforcement-learning, Wildfire Management, showcase
\item \textbf{Arxiv:}  (if any)
\end{itemize}
%todo: type (conference, journal, ptatne...)
%todo: can we check if a field is empy?
%todo: copy all paperdesc text from cv to bibtex in this way
%todo: create another version of this template for markdown, with a page separator

\fbox{%
\parbox{\textwidth}{%
        \textbf{CITATION: \texttt{Subramanian2017} \cite{Subramanian2017}}:\\
            Sriram Ganapathi Subramanian and Mark Crowley.  ``Learning Forest Wildfire Dynamics from Satellite Images Using Reinforcement Learning'', \emph{}. In \emph{Conference on Reinforcement Learning and Decision Making}. (), 244-248 pages. Ann Arbor, MI, USA., 2017.
    }%
}

\textbf{Abstract:} 

\textbf{paperdesc:} 

\textbf{Description:} 



\newpage
\subsubsection{\textbf{-2016} : Anomaly Detection Using Inter-Arrival Curves for Real-time Systems}
%salem-ecrts-2016
%status: 1
\begin{itemize}
\item \textbf{keywords:} anomaly detection,arrival curves,embedded systems,proj-MassiveTraceAnomalyDetection,trace analysis, showcase
\item \textbf{Arxiv:}  (if any)
\end{itemize}
%todo: type (conference, journal, ptatne...)
%todo: can we check if a field is empy?
%todo: copy all paperdesc text from cv to bibtex in this way
%todo: create another version of this template for markdown, with a page separator

\fbox{%
\parbox{\textwidth}{%
        \textbf{CITATION: \texttt{salem-ecrts-2016} \cite{salem-ecrts-2016}}:\\
            Mahmoud Salem, Mark Crowley, and Sebastian Fischmeister.  ``Anomaly Detection Using Inter-Arrival Curves for Real-time Systems'', \emph{}. In \emph{2016 28th Euromicro Conference on Real-Time Systems}. (), 97--106 pages. Toulouse, France, 2016.
    }%
}

\textbf{Abstract:} ---Real-time embedded systems are a significant class of applications, poised to grow even further as automated vehicles and the Internet of Things become a reality. An important problem for these systems is to detect anomalies during operation. Anomaly detection is a form of classification, which can be driven by data collected from the system at execution time. We propose inter-arrival curves as a novel analytic modelling technique for discrete event traces. Our approach relates to the existing technique of arrival curves and expands the technique to anomaly detection. Inter-arrival curves analyze the behaviour of events within a trace by providing upper and lower bounds to their inter-arrival occurrence. We exploit inter-arrival curves in a classification framework that detects deviations within these bounds for anomaly detection. Also, we show how inter-arrival curves act as good features to extract recurrent behaviour that these systems often exhibit. We demonstrate the feasibility and viability of the fully implemented approach with an industrial automotive case study (CAN traces) as well as a deployed aerospace case study (RTOS kernel traces).

\textbf{paperdesc:} 

\textbf{Description:} 



\newpage
\subsubsection{\textbf{-2015} : PAC Optimal MDP Planning with Application to Invasive Species Management}
%Taleghan2015
%status: 1
\begin{itemize}
\item \textbf{keywords:} Good-Turing,MDP,Markov-decision-processes,Computational-Sustainability,grant-wici16,invasive species management,Machine Learning,MDP,optimization,Reinforcement-Learning, showcase
\item \textbf{Arxiv:}  (if any)
\end{itemize}
%todo: type (conference, journal, ptatne...)
%todo: can we check if a field is empy?
%todo: copy all paperdesc text from cv to bibtex in this way
%todo: create another version of this template for markdown, with a page separator

\fbox{%
\parbox{\textwidth}{%
        \textbf{CITATION: \texttt{Taleghan2015} \cite{Taleghan2015}}:\\
            Majid Alkaee Taleghan, Thomas G. Dietterich, Mark Crowley, Kim Hall, and H. Jo Albers.  ``PAC Optimal MDP Planning with Application to Invasive Species Management'', \emph{Journal of Machine Learning Research}. In \emph{}. 16(), 3877--3903 pages. , 2015.
    }%
}

\textbf{Abstract:} In a simulator-defined MDP, the Markovian dynamics and rewards are provided in the form of a simulator from which samples can be drawn. This paper studies MDP planning algorithms that attempt to minimize the number of simulator calls before terminating and outputting a policy that is approximately optimal with high probability. The paper introduces two heuristics for efficient exploration and an improved confidence interval that enables earlier termination with probabilistic guarantees. We prove that the heuristics and the confidence interval are sound and produce with high probability an approximately optimal policy in polynomial time. Experiments on two benchmark problems and two instances of an invasive species management problem show that the improved confidence intervals and the new search heuristics yield reductions of between 8{\%} and 47{\%} in the number of simulator calls required to reach near-optimal policies.

\textbf{paperdesc:} This further expands on results from [DietterichAAAI2013], again I was a central contributor working through the theory of this paper with Prof. Dietterich and coordinating experimental results with his PhD student Taleghan. Hall and Albers were the domain application experts.

\textbf{Description:} 



\newpage
\subsubsection{\textbf{-2014} : Using equilibrium policy gradients for spatiotemporal planning in forest ecosystem management}
%crowleyieee2013
%status: 1
\begin{itemize}
\item \textbf{keywords:} Computational-Sustainability,Ecosystem management,Forest Management,Machine Learning,Markov decision processes,Optimization,Policy gradient planning,Reinforcement-Learning,grant-wici16,proj-spatiallyspreadingprocess,spatiotemporal planning, showcase
\item \textbf{Arxiv:}  (if any)
\end{itemize}
%todo: type (conference, journal, ptatne...)
%todo: can we check if a field is empy?
%todo: copy all paperdesc text from cv to bibtex in this way
%todo: create another version of this template for markdown, with a page separator

\fbox{%
\parbox{\textwidth}{%
        \textbf{CITATION: \texttt{crowleyieee2013} \cite{crowleyieee2013}}:\\
            Mark Crowley.  ``Using equilibrium policy gradients for spatiotemporal planning in forest ecosystem management'', \emph{IEEE Transactions on Computers}. In \emph{}. 63(1), 142--154 pages. , 2014.
    }%
}

\textbf{Abstract:} Spatiotemporal planning involves making choices at multiple locations in space over some planning horizon to maximize utility and satisfy various constraints. In Forest Ecosystem Management, the problem is to choose actions for thousands of locations each year including harvesting, treating trees for fire or pests, or doing nothing. The utility models could place value on sale of lumber, ecosystem sustainability or employment levels and incorporate legal and logistical constraints on actions such as avoiding large contiguous areas of clearcutting. Simulators developed by forestry researchers provide detailed dynamics but are generally inaccesible black boxes. We model spatiotemporal planning as a factored Markov decision process and present a policy gradient planning algorithm to optimize a stochastic spatial policy using simulated dynamics. It is common in environmental and resource planning to have actions at different locations be spatially interelated, this makes representation and planning challenging. We define a global spatial policy in terms of interacting local policies defining distributions over actions at each location conditioned on actions at nearby locations. Markov chain Monte Carlo simulation is used to sample landscape policies and estimate their gradients. Evaluation is carried out on a forestry planning problem with 1,880 locations using a variety of value models and constraints. Index

\textbf{paperdesc:} 

\textbf{Description:} 



\newpage
\subsubsection{\textbf{-2013} : Allowing a wildfire to burn: Estimating the effect on future fire suppression costs}
%houtman2013
%status: 1
\begin{itemize}
\item \textbf{keywords:} bio-economic modelling,Wildfire-Management,Wildfire-Management,grant-wici16,proj-spatiallyspreadingprocess,spatial simulation,Wildfire-Management, showcase
\item \textbf{Arxiv:}  (if any)
\end{itemize}
%todo: type (conference, journal, ptatne...)
%todo: can we check if a field is empy?
%todo: copy all paperdesc text from cv to bibtex in this way
%todo: create another version of this template for markdown, with a page separator

\fbox{%
\parbox{\textwidth}{%
        \textbf{CITATION: \texttt{houtman2013} \cite{houtman2013}}:\\
            Rachel M. Houtman, Claire A. Montgomery, Aaron R. Gagnon, David E. Calkin, Thomas G. Dietterich, Sean McGregor, and Mark Crowley.  ``Allowing a wildfire to burn: Estimating the effect on future fire suppression costs'', \emph{International Journal of Wildland Fire}. In \emph{}. 22(7), 871--882 pages. , 2013.
    }%
}

\textbf{Abstract:} Where a legacy of aggressive wildland fire suppression has left forests in need of fuel reduction, allowing wildland fire to burn may provide fuel treatment benefits, thereby reducing suppression costs from subsequent fires. The least-cost-plus-net-value-change model of wildland fire economics includes benefits of wildfire in a framework for evaluating suppression options. In this study, we estimated one component of that benefit -- the expected present value of the reduction in suppression costs for subsequent fires arising from the fuel treatment effect of a current fire. To that end, we employed Monte Carlo methods to generate a set of scenarios for subsequent fire ignition and weather events, which are referred to as sample paths, for a study area in central Oregon. We simulated fire on the landscape over a 100-year time horizon using existing models of fire behaviour, vegetation and fuels development, and suppression effectiveness, and we estimated suppression costs using an existing suppression cost model. Our estimates suggest that the potential cost savings may be substantial. Further research is needed to estimate the full least-cost-plus-net-value-change model. This line of research will extend the set of tools available for developing wildfire management plans for forested landscapes.

\textbf{paperdesc:} This work formed the core MSc research of Houtman, I managed collaboration amongst the graduate students, guiding their experimental work and results. I also worked on most of the writing on data analysis and editing the final drafts in general. This paper has been cited frequently as the issue of dealing with forest wildfire grows in importance.

\textbf{Description:} NULL



\newpage
\subsubsection{\textbf{-2013} : PAC Optimal Planning for Invasive Species Management: Improved Exploration for Reinforcement Learning from Simulator-Defined MDPs}
%Dietterich2013
%status: 1
\begin{itemize}
\item \textbf{keywords:} PAC learning,Computational-Sustainability,invasive species management,Machine Learning,mdp,planning, showcase
\item \textbf{Arxiv:}  (if any)
\end{itemize}
%todo: type (conference, journal, ptatne...)
%todo: can we check if a field is empy?
%todo: copy all paperdesc text from cv to bibtex in this way
%todo: create another version of this template for markdown, with a page separator

\fbox{%
\parbox{\textwidth}{%
        \textbf{CITATION: \texttt{Dietterich2013} \cite{Dietterich2013}}:\\
            Thomas G Dietterich, Majid Alkaee Taleghan, and Mark Crowley.  ``PAC Optimal Planning for Invasive Species Management: Improved Exploration for Reinforcement Learning from Simulator-Defined MDPs'', \emph{}. In \emph{Proceedings of the AAAI Conference on Artificial Intelligence (AAAI-2013)}. (), 7 pages. Bellevue, WA, USA, 2013.
    }%
}

\textbf{Abstract:} Often the most practical way to define a Markov Decision Process (MDP) is as a simulator that, given a state and an action, produces a resulting state and immediate reward sampled from the corresponding distributions. Simulators in natural resource management can be very expensive to execute, so that the time required to solve such MDPs is dominated by the number of calls to the simulator. This paper presents an algorithm, DDV, that combines improved confidence intervals on the Q values (as in interval estimation) with a novel upper bound on the discounted state occupancy probabilities to intelligently choose state-action pairs to explore. We prove that this algorithm terminates with a policy whose value is within $\epsilon$ of the optimal policy (with probability 1-$\delta$) after making only polynomially many calls to the simulator. Experiments on one benchmark MDP and on an MDP for invasive species management show very large reductions in the number of simulator calls required.

\textbf{paperdesc:} 

\textbf{Description:} 



\newpage
\subsubsection{\textbf{-2013} : Cyclic causal models with discrete variables: Markov chain equilibrium semantics and sample ordering}
%PooleCrowley2013
%status: 1
\begin{itemize}
\item \textbf{keywords:} Causality,probabilistic inference, showcase
\item \textbf{Arxiv:}  (if any)
\end{itemize}
%todo: type (conference, journal, ptatne...)
%todo: can we check if a field is empy?
%todo: copy all paperdesc text from cv to bibtex in this way
%todo: create another version of this template for markdown, with a page separator

\fbox{%
\parbox{\textwidth}{%
        \textbf{CITATION: \texttt{PooleCrowley2013} \cite{PooleCrowley2013}}:\\
            David Poole and Mark Crowley.  ``Cyclic causal models with discrete variables: Markov chain equilibrium semantics and sample ordering'', \emph{}. In \emph{IJCAI International Joint Conference on Artificial Intelligence}. (), 1060--1068 pages. Beijing, China, 2013.
    }%
}

\textbf{Abstract:} We analyze the foundations of cyclic causal models for discrete variables, and compare structural equation models (SEMs) to an alternative semantics as the equilibrium (stationary) distribution of a Markov chain. We show under general conditions, discrete cyclic SEMs cannot have independent noise, even in the simplest case, cyclic structural equation models imply constraints on the noise. We give a formalization of an alternative Markov chain equilibrium semantics which requires not only the causal graph, but also a sample order. We show how the resulting equilibrium is a function of the sample ordering, both theoretically and empirically.

\textbf{paperdesc:} 

\textbf{Description:} 











\newpage

\bibliographystyle{plainnat}
\bibliography{papers}





\end{document}
